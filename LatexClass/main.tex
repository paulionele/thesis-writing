%This 'main' document will serve as the root document to draw the whole document together.
%Compile this document in order to 
%Comments:
%--in order to specify the relative file path to the introduction document, need \subimport{}{}.
\documentclass{jthesis}
%\usepackage{algorithm}

\usepackage[ruled,vlined]{algorithm2e}

%\usepackage{algpseudocode}% http://ctan.org/pkg/algorithmicx
%\newcommand{\var}[1]{{\ttfamily#1}}% variable

\usepackage{import}
\graphicspath{ {../images/} }

\title{Simulation of Diffusion in Heterogeneous Media}
\date{Final draft submitted -- April 24, 2016}
\author{Paul Ionele}

\begin{document}

\pagenumbering{roman} 
\maketitle

\makeabstract{
	In this thesis project, we created 1D and 2D heterogeneous simple cell and tissue models and simulated unbiased diffusion of non-interacting particles in these systems using lattice Monte Carlo and lattice master equation methods. The models basically consisted of low diffusivity cellular regions and high diffusivity extracellular regions, separated by semi-permeable boundaries.	The first goal was to develop two computational methods that can be more easily implemented to simulate diffusion, compared to some analytical methods. The last goal was to gain insight into the various diffusive behaviours occurring in the system and the long-time effective diffusivity. This was done by changing the semi-permeable boundary transition probabilities, region diffusivities, and system dimensionality and geometry. It was found that for the 1D heterogeneous systems, sub-diffusive behaviour dominated super-diffusive behaviour and the long-time effective diffusivity depends strongly on the boundary transition probabilities and region diffusivities. For the 2D heterogeneous systems, the presence of an extracellular channel that allows particles to pass through length of the system unobstructed, results in diffusive behaviour and effective diffusivities very different from the 1D systems. Super-diffusive behaviour was more prominent and long-time effective diffusivities were increased overall. Lastly, the long-time effective diffusivity was practically unaffected by changes in the cellular region diffusivity.
}

%\makeacknowledgements{
%Thank some people that you like here.
%}

\makedeclaration

\maketableofcontents

\pagenumbering{arabic}
\doublespacing

% % % % % % % % % % % % % % % % % % % % % % % % % % % % % %
%Include the different document sections in the space below!
\subimport{../chapters/}{introduction}
\newpage
\subimport{../chapters/}{methods}
\newpage
\subimport{../chapters/}{discussion}
\newpage
\subimport{../chapters/}{conclusion}

% % % % % % % % % % % % % % % % % % % % % % % % % % % % % %
\begin{thebibliography}{}

\bibitem[Campbell, Reece (2008)]{cr-biology} Campbell, N.A., Reece, J.B., et al. 2008. Biology. 8th ed. Pearson Benjamin Cummings.

\bibitem[Patton, Thibodeau (2013)]{ap} Patton K.T., Thibodeau G.A.. 2013. Anatomy and Physiology. 8th ed. Elsevier.

\bibitem[J. Philibert (2006)]{diffusion-1} One and a Half Century of Diffusion: Fick, Einstein, Before and Beyond. Jean Philibert, diffusion-fundamentals.org \textbf{4} (2006) 6, pp 6.1-6.19.

\bibitem[Haan, Chubynsky, \& Slater (2012)]{haan} Monte Carlo Approaches for Simulating a Particle at a Diffusivity Interface and the ``Ito--Stratonovich Dilemma". de Haan, H.W., Chubynsky, M.V., Slater, G.W. (2012) arXiv:1208.5081

\bibitem[Stickler, Schachinger (2014)]{cp} Stickler, B.A., Schachinger, E.. 2014. Basic Concepts in Computational Physics. Springer.

\bibitem[J. Cervi (2015)]{cervi} A Numerical Study of the Effects of Inhomogeneous Media in Diffusion Weighted Imaging. Jessica Cervi. 2015. UOIT MSc. thesis.

\bibitem[Toral, Colet (2014)]{master-equations} Toral, R., Colet, P.. 2014. Stochastic Numerical Methods: An Introduction for Students and Scientists. Wiley.

\end{thebibliography}

%\appendix

%\chapter{Appendix}
%\label{app:sample}
%This is just an example of an appendix.

\end{document}
