NOTES

On the topic of fluid visocity and cell composition:

Most of the cytosol is water, which makes up about 70\% of the total volume of a typical cell.[11] The pH of the intracellular fluid is 7.4.[12] while human cytosolic pH ranges between 7.0 - 7.4, and is usually higher if a cell is growing.[13] The viscosity of cytoplasm is roughly the same as pure water, although diffusion of small molecules through this liquid is about fourfold slower than in pure water, due mostly to collisions with the large numbers of macromolecules in the cytosol.[14] Studies in the brine shrimp have examined how water affects cell functions; these saw that a 20\% reduction in the amount of water in a cell inhibits metabolism, with metabolism decreasing progressively as the cell dries out and all metabolic activity halting when the water level reaches 70\% below normal.[2]

Luby-Phelps K (2000). "Cytoarchitecture and physical properties of cytoplasm: volume, viscosity, diffusion, intracellular surface area" (PDF). Int. Rev. Cytol. International Review of Cytology 192: 189–221. doi:10.1016/S0074-7696(08)60527-6. ISBN 978-0-12-364596-8. PMID 10553280.[11]

Verkman AS (January 2002). "Solute and macromolecule diffusion in cellular aqueous compartments". Trends Biochem. Sci. 27 (1): 27–33. doi:10.1016/S0968-0004(01)02003-5. PMID 11796221.