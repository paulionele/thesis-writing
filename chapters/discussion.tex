\chapter{Results and Analysis}
\label{chapter:results-analysis}
	
	In this chapter, we begin an analysis of 1D heterogeneous systems, then move our attention to 2D heterogeneous systems. Although 1D homogeneous systems were simulated, they did not provide any interesting or new insight into diffusive behaviour, therefore those simulations and analytical data are not presented in detail. We begin each section with selected images of simulated density distributions to motivate further analysis. More specifically, we analyze how changing the boundary transition probabilities, region diffusivities, and cell geometry (2D only), affects the mean-squared-displacement (MSD) and show that sub- and super-diffusive regimes exist. Lastly, we show that the long-time effective diffusivity is non-trivial even for simple-cell geometries.
	
\section{Preparation for Analysis}
	Since the same tools will be used to analyze similar plots for the 1D and 2D systems, this section serves to outline fundamentals that will be referred to and utilized frequently in the analysis.
	
	The analytical data computed from the density distribution at every time step included the MSD: $ \langle \Delta x^2 \rangle = \langle x^2 \rangle - \langle x \rangle^2 $. Einstein's diffusion equation says that for steady/regular diffusion in one dimension, the following relationship is valid: $ \langle \Delta x^2 \rangle = 2D \cdot t $. Essentially, in the case of steady diffusion, it is expected that the MSD increases linearly with time and furthermore, the log-log plot of such a relation has a slope of unity, since $ t $ is raised to the power of 1. For most systems studied, there was some non-linearity in time indicating the presence of anomalous diffusion. We introduce a modified Einstein's diffusion equation with an additional exponential factor $ \beta $.
	
	\begin{equation}
	\label{eq:einstein-modified}
		\langle \Delta x^2 \rangle = 2D \cdot t^{\beta}
	\end{equation}
	
	This factor $ \beta $ can be used to determine and quantify the diffusion regime of the system at some time $ t $. Taking the base-10 logarithm of both sides of Equation \ref{eq:einstein-modified}:

	\begin{equation}
		\log\left( \langle \Delta x^2 \rangle \right) = \beta \cdot \log(t) + \log(2D)
	\end{equation}	
	
	On a linear plot, this appears as a relation of the form $ Y = \beta X + C $ and if $ \beta = 1 $, then the diffusion process is steady. It is possible to determine what $ \beta $ is as a function of time. If we take the derivative $ dY/dX $, then:
	
	\begin{equation}
		\dfrac{dY}{dX} = \beta (t)
	\end{equation}
	
	A plot of $ \beta (t) $ versus $ t $ reveals the magnitude of the non-linearity and identifies if the diffusion process occurring sub- or super-diffusive at a particular time. For a sub-diffusive process $ \beta < 1 $, for a super-diffusive process $ \beta > 1 $, and for steady diffusion $ \beta = 1 $.
	
	Another analytical quantity of interest was the effective diffusion coefficient $ D_\textrm{eff} $. The effective diffusion at any point in time may be calculated from Einstein's diffusion equation.
	
	\begin{equation}
		D_\textrm{eff} = \dfrac{\langle \Delta x^2 \rangle}{2t}
	\end{equation}
	
	Plots of $ D_\textrm{eff} $ versus $ t $ may be constructed and are useful in determining how the long-time effetcive diffusivity depends on the semi-permeable boundary transition probability, ratio of cellular to extracellular diffusivities, and system geometry (2D).
	 
	
\section{1D Systems}
\label{sec:ra-1D}
	For very large systems or short simulation times---conditions where the density distribution is zero at the absolute boundaries---agreement between the Monte Carlo (MC) and master equation (ME) simulation solutions, and analytical solutions (\ref{eq:1D-diffusion-equation-solution}) of the 1D diffusion equation with no net drift and constant diffusion factor (\ref{eq:1D-diffusion-equation}), was observed (Figure \ref{fig:11U_homogeneous_plots_1D}). 

	\begin{figure}[h]
		\centering
		\includegraphics[width=1.0\linewidth]{../images/1D/11U_homogeneous_plots_1D}
		\caption{Simple 1D system with no internal semi-permeable boundaries. Simulation solutions plotted for ME and MC methods. Additionally, an exact solution to diffusion equation (Fokker-Planck) is plotted. Parameter specifications: 330 lattice sites, $ D = 0.1 $, $ N = 8\cdot10^5 $ particles in MC simulation, and density normalized.}
		\label{fig:11U_homogeneous_plots_1D}
	\end{figure}
	
	\nointerlineskip The 1D diffusion equation \ref{eq:1D-diffusion-equation} may be known as a `simple' Fokker-Planck or Smoluchowski equation and is sometimes utilized as a model of classical Brownian motion.
	
	\begin{equation}
	\label{eq:1D-diffusion-equation}
		\frac{\partial\rho(x,t)}{\partial t} = D_0 \frac{\partial^2 \rho(x,t)}{\partial x^2}
	\end{equation}
	
	Assuming that $ N $ particles start from the origin, Equation \ref{eq:1D-diffusion-equation} has the solution:
	
	\begin{equation}
	\label{eq:1D-diffusion-equation-solution}
		\rho(x,t)=\frac{N}{\sqrt{4\pi Dt}}e^{-\frac{x^2}{4Dt}}
	\end{equation}	
	
	A variant of Equation \ref{eq:1D-diffusion-equation-solution} was used to create the Fokker-Planck plot in Figure \ref{fig:11U_homogeneous_plots_1D}. The solution to \ref{eq:1D-diffusion-equation} is not a solution to a boundary value problem. For example, it is possible to solve \ref{eq:1D-diffusion-equation} for reflecting absolute boundary conditions: $ \rho_x(0,t) = \rho_x(L,t) = 0 $. It is even possible to solve it for `repeating' boundary conditions that represent semi-permeable internal boundaries (Jessica Cervi's thesis); these specific problems were not explored and are not discussed further. However, it is comparatively easy to obtain a solution to such a system with internal semi-permeable and absolute reflecting boundaries by utilizing MC and ME methods for the simulation of the system; Figure \ref{fig:11U_heterogeneous_plots_1D} shows such a solution.
	
	With reference to Figure \ref{fig:11U_homogeneous_plots_1D}, note the good agreement between the ME and MC simulation solutions and analytical solution. The ME simulation solution and analytical solution are perfectly superimposed. The MC simulation shows characteristic statistical fluctuations that reflect the underlying `random' movement of individual particles in the simulation. The relative magnitude of these fluctuations decreases with an increasing number of particles $ N $ according to $ 1/\sqrt{N} $.

	\begin{figure}[h]
		\centering
		\includegraphics[width=1.0\linewidth]{../images/1D/11U_heterogeneous_plots_1D}
		\caption{1D heterogeneous system with internal semi-permeable boundaries. Simulation solutions plotted for ME and MC methods. Parameter specifications: 330 lattice sites, 15 lattice sites per cellular or extracellular region, $ D_i = 0.05 $, $ D_e = 0.2 $, and $ P_{i\rightarrow e} = 0.05 $. $ N = 5 \cdot 10^5 $ particles in the MC simulation and density normalized. Vertical dashed lines show semi-permeable boundary positions.}
		\label{fig:11U_heterogeneous_plots_1D}
	\end{figure}
	
	Statistical fluctuations in the MC simulation solution are more visible in Figure \ref{fig:11U_heterogeneous_plots_1D}. The ME simulation was orders of magnitude faster in reaching completion and does not show any fluctuations since the density distribution is evolved in time under the assumption that there are \textsl{very} many particles at each lattice site. It is expected that in the limit of an infinite number of particles, the MC solution converges to the ME solution. Comparisons to analytical or numerical solutions of the Fokker-Planck equation were not made but doing so may be of interest in the future.
	
	\begin{figure}[h]
		\centering
		\includegraphics[width=1.0\linewidth]{../images/1D/3U_heterogeneous_plots_1D}
		\caption{1D heterogeneous system with internal semi-permeable boundaries. Simulation solutions plotted for ME and MC methods. Parameter specifications: 90 lattice sites, 15 lattice sites per cellular or extracellular region, $ D_i = 0.05 $, $ D_e = 0.2 $, and $ P_{i\rightarrow e} = 0.01 $. $ N = 8 \cdot 10^5 $ particles in the MC simulation and density normalized. Vertical dashed lines show semi-permeable boundary positions.}
		\label{fig:3U_heterogeneous_plots_1D}
	\end{figure}
	
	\newpage
	Interesting to note in Figure \ref{fig:11U_heterogeneous_plots_1D} and \ref{fig:3U_heterogeneous_plots_1D}, the magnitude of the fluctuations in the MC solution appear larger at higher densities and smaller at lower densities. In Figure \ref{fig:11U_heterogeneous_plots_1D}, extracellular regions with 4-times larger diffusivity equilibrate faster than the cellular regions; this is visible as `flat' versus `sloped' density distributions. Figure \ref{fig:3U_heterogeneous_plots_1D} shows behaviour at the absolute reflecting boundaries.
	
	\begin{figure}[h]
		\centering
		\includegraphics[width=1.0\linewidth]{../images/1D/11U_heterogeneous_plots_1D_nonphysical}
		\caption{1D heterogeneous system with internal semi-permeable boundaries. Non-physical density distribution simulated by violating the semi-permeable boundary condition (sometimes known as the isothermal boundary condition). Parameter specifications: 330 lattice sites, 15 lattice sites per cellular or extracellular region, $ D_i = 0.05 $, $ D_e = 0.2 $, and $ P_{i\rightarrow e} = 0.01 $ and $ P_{e\rightarrow i} =  0.005 $. Correctly chosen, $ P_{e\rightarrow i} =  0.0025 $. Vertical dashed lines show semi-permeable boundary positions. }
		\label{fig:11U_heterogeneous_plots_1D_nonphysical}
	\end{figure}
	
	\newpage
	Figure \ref{fig:11U_heterogeneous_plots_1D_nonphysical} shows the necessity of having coupled transition probabilities for the semi-permeable boundaries. The distribution in the long-time limit does not approach a uniform distribution, as would be expected for any system with passive semi-permeable boundaries and reflecting absolute boundaries. In the long-time, the density grows in certain cells and shrinks in others, similar to what might be expected if the semi-permeable boundaries behaved in an active-selective manner. 
	
	\newpage
	Section \ref{sec:1D-boundary-transition-probability} explores the effects of changing the semi-permeable boundary transition probability while maintaining region diffusivities constant. The second and last section \ref{sec:1D-cellular-extracellular-diffusivities} explores the effects of changing the ratio of the cellular and extracellular (C/EC) diffusivities while maintain the semi-permeable boundary transition probability constant. In both sections and for all simulations, the size of the unit cell was constant, and C/EC regions were of the same dimensions.
	
\subsection{Boundary Transition Probability}
\label{sec:1D-boundary-transition-probability}
	The total transition probability is the product of two probabilities; the probability that a particle steps towards a boundary and the transition probability. The transition probability $ P_{i \rightarrow e} $ (``pie" in figure) is specified but the corresponding probability  $ P_{e \rightarrow i} $ can be easily calculated since the probabilities are coupled. For each plot, a different transition probability tested while all other parameters were held constant.
	
	\begin{figure}[h!]
		\centering
		\includegraphics[width=1.0\linewidth]{../images/1D/pie_msd_1D}
		\caption{MSD versus time for 1D heterogeneous system. Parameter specifications: 3330 lattice sites, 15 lattice sites per cellular or extracellular region, $ D_i = 0.05 $, $ D_e = 0.2 $, and simulation time is $ 7\cdot 10^5 $ time steps. $ D = 0.2 $ for top dashed line, $ D = 0.05 $ for middle dashed line, and $ D = 0.5 \cdot (P_{i \rightarrow e} + P_{e \rightarrow i}) $ for the bottom dashed line (an average of the transition probabilities).}
		\label{fig:pie_msd_1D}
	\end{figure}
	
	 In Figure \ref{fig:pie_msd_1D}, the slope is clearly not unity for all time, for most of the plots, indicating the presence of anomalous diffusion. With decreasing transition probability, the magnitude of deviation from steady diffusion increases. Interestingly, for the largest transition probability simulated ($ P_{i \rightarrow e} = 0.150 $), the magnitude of deviation from steady diffusion is very small. It may be possible that for large transition probabilities, as long as the semi-permeable boundary transition condition is respected, approximately steady diffusion occurs for all time. In the long-time, the diffusion process becomes steady again for all transition probabilities tested. 
	 
	 In Figure \ref{fig:pie_beta_deff_1D}, the extent of the anomalous diffusion is clear in the $ \beta (t) $ versus $ t $ top figure. For all transition probabilities, steady diffusion occurs from the start of the simulation to some time where particles meet the semi-permeable boundary; this occurs at the same time for all simulations since C/EC diffusivity was constant. Both the magnitude of sub-diffusive behaviour and length of time before reaching steady diffusion, increase with decreasing transition probability. Super-diffusive behaviour appears only for two transition probabilities; it might be interesting to find the threshold transition probability that results in super-diffusive behaviour.
	 
	 In the bottom figure, the effective diffusivity approaches an average of the boundary transition probabilities, at least for transition probabilities much smaller than C/EC diffusivities. This is shown for $ P_{i \rightarrow e} = 0.150 $ and is observed for the other transition probabilities as well; however, this `average approach' sometimes under- or over-estimates the effective diffusivity obtained in the simulations and does not work well for transition probabilities approaching C/EC diffusivities, which may indicate dependence on regional diffusivities.
	
	\begin{figure}[h!]
		\centering
		\includegraphics[width=1.0\linewidth]{../images/1D/pie_beta_deff_1D}
		\caption{$ \beta (t) $ versus log(time) and $ D_\textrm{eff} $ versus time for 1D heterogeneous system. Parameter specifications: 3330 lattice sites, 15 lattice sites per cellular or extracellular region, $ D_i = 0.05 $, $ D_e = 0.2 $, and simulation time is $ 7\cdot 10^5 $ time steps.}
		\label{fig:pie_beta_deff_1D}
	\end{figure}

\newpage
\subsection{Cellular and Extracellular Diffusivities}
\label{sec:1D-cellular-extracellular-diffusivities}
	For each plot, the ratio of cellular to extracellular diffusivities (indicated as pi/pe on the figures), while all other parameters were held constant. More specifically, the cellular diffusivity was varied while the extracellular diffusivity was constant.
	
	\begin{figure}[h]
		\centering
		\includegraphics[width=1.0\linewidth]{../images/1D/pipe_msd_1D}
		\caption{}
		\label{fig:pipe_msd_1D}
	\end{figure}
	
	\begin{figure}[h]
		\centering
		\includegraphics[width=1.0\linewidth]{../images/1D/pipe_beta_deff_1D}
		\caption{}
		\label{fig:pipe_beta_deff_1D}
	\end{figure}

\newpage
\section{2D Systems}
\label{sec:ra-2D}

	\begin{figure}[h]
		\centering
		\includegraphics[width=1.0\linewidth]{../images/2D/heterogeneous_3U_2D}
		\caption{}
		\label{fig:heterogeneous_3U_2D}
	\end{figure}
	
	\begin{figure}[h]
		\centering
		\includegraphics[width=1.0\linewidth]{../images/2D/MC_heterogeneous_3U_2D}
		\caption{}
		\label{fig:MC_heterogeneous_3U_2D}
	\end{figure}
	
	\begin{figure}[h]
		\centering
		\includegraphics[width=1.0\linewidth]{../images/2D/heterogeneous_11U_2D}
		\caption{}
		\label{fig:heterogeneous_11U_2D}
	\end{figure}
	
	\begin{figure}[h]
		\centering
		\includegraphics[width=1.0\linewidth]{../images/2D/error_in_diffusion}
		\caption{}
		\label{fig:error_in_diffusion}
	\end{figure}
	
	\begin{figure}[h]
		\centering
		\includegraphics[width=1.0\linewidth]{../images/2D/pie_msd_2D}
		\caption{}
		\label{fig:pie_msd_2D}
	\end{figure}
	
	\begin{figure}[h]
		\centering
		\includegraphics[width=1.0\linewidth]{../images/2D/pie_beta_deff_2D}
		\caption{}
		\label{fig:pie_beta_deff_2D}
	\end{figure}
	
	\begin{figure}[h]
		\centering
		\includegraphics[width=1.0\linewidth]{../images/2D/pipe_msd_2D}
		\caption{}
		\label{fig:pipe_msd_2D}
	\end{figure}
	
	\begin{figure}[h]
		\centering
		\includegraphics[width=1.0\linewidth]{../images/2D/pipe_beta_deff_2D}
		\caption{}
		\label{fig:pipe_beta_deff_2D}
	\end{figure}
	
	\begin{figure}[h]
		\centering
		\includegraphics[width=1.0\linewidth]{../images/2D/ye_msd_2D}
		\caption{}
		\label{fig:ye_msd_2D}
	\end{figure}
	
	\begin{figure}[h]
		\centering
		\includegraphics[width=1.0\linewidth]{../images/2D/ye_beta_deff_2D}
		\caption{}
		\label{fig:ye_beta_deff_2D}
	\end{figure}

	