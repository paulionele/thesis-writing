\chapter{Conclusion}
	
	In this thesis project, we created heterogeneous 1D and 2D simple cell and tissue models and simulated unbiased diffusion of non-interacting particles in these systems using lattice Monte Carlo (MC) and lattice master equation (ME) methods. 
	
	There were two main goals in this project. The first goal was to develop two computational methods that could be used as alternatives to simulate diffusion in systems that might otherwise be very difficult to simulate by solving the diffusion equation (with the appropriate boundary conditions). The MC method used, sometimes known as a particle-based method, can provide particle trajectory information, but is relatively slow and computed density distributions suffer from statistical fluctuations. The ME method used was a distribution-based approach; the particle density was evolved in time. Its evolution represented the average behaviour of infinitely many particles and therefore it can be said that with an increasing number of particles used, the MC method approaches the ME method. The ME method is unable to provide particle trajectory information but ME simulations are much faster to compute because computations are performed on lattice-site basis and not on every individual particle as in the MC simulations. The second goal was to perform various analyses on simple 1D and 2D heterogeneous systems built from repeating units called unit cells. More specifically, we analyzed how changing different parameters such as the: semi-permeable boundary transition probabilities, region diffusivities, and cell geometry (2D only), affected the diffusion behaviour and long-time effective diffusivity. When a specific parameter was tested, all other parameters were held constant.
	
	\newpage
	
	All heterogeneous systems were characterized by equal-sized regions of different diffusivities repeating in series and separated by semi-permeable boundaries. The systems as a whole were enclosed by reflecting absolute boundaries. Since diffusion was unbiased, it was expected that the long-time particle density distribution is equal everywhere. In order for that to occur, it was necessary to use a semi-permeable boundary condition called the ``iso-thermal" condition, which can be derived using detailed balance theory. This boundary condition specifically relates/couples the boundary transition probabilities between regions of different diffusivities and depends on those diffusivities.
	
	For all heterogeneous systems studied, normal and sub- and super-diffusive behaviour was observed. Normal diffusive behaviour was characterized by a computed mean-squared-displacement (MSD) that scaled linearly in time ($ \textrm{MSD} \propto t $). For sub-diffusive behaviour, $ \textrm{MSD} \propto t^{\beta < 1} $, and for super-diffusive behaviour, $ \textrm{MSD} \propto t^{\beta > 1} $. In general, the semi-permeable boundary acts to impede or slow particle diffusion, therefore causing temporary sub-diffusive behaviour as particles begin to interact with a boundary. If the region outside the starting region of the particles is of higher diffusivity, super-diffusive behaviour follows after sub-diffusive behaviour. In the long-time limit, diffusive behaviour returns to normal.
	
	...
	Beginning with the 1D heterogeneous systems, and have shown that sub- and super-diffusive behaviours exist, although the magnitude of super-diffusive behaviour is smaller than sub-diffusive behaviour.
	
	...
	
	In the end, some `future work' remains. First, a correction factor must be determined and applied to the relation used to calculate the effective diffusivity in the 2D systems. The MSD calculation used to determine the effective diffusivity only used the x-position of particles since the systems studied were much larger in $ \hat{x} $ than in $ \hat{y} $. This means that for some time steps, the calculated MSD was no different than the previous calculation since no movement occurred in $ \hat{x} $, but movement did occur in $ \hat{y} $. Second, simulation time and length needs to be mapped to physical time and length. Lastly, parameter space for the simulations needs to be explored further; the diffusive behaviours are well understood but the calculated effective diffusivity in the long-time and its exact dependence on different parameters mostly remains a mystery.