\chapter{Conclusion}
	
	In this thesis project, we created 1D and 2D simple cell and tissue heterogeneous models and simulated unbiased diffusion in these systems using Monte Carlo and master equation methods. The goals were to develop two computational methods for the simulation of diffusion, that can be easily implemented, and to gain insight into the various diffusive behaviours occurring in the system and calculate the long-time effective diffusivity for various cell parameters. We analyzed how changing the boundary transition probabilities, region diffusivities, and cell geometry (2D only), affected the mean-squared-displacement and have shown that sub- and super-diffusive behaviours exist. Lastly, we have shown that the long-time effective diffusivity is non-trivial even for simple-cell geometries.
	
	In the end, some `future work' remains. First, a correction factor must be determined and applied to the relation used to calculate the effective diffusivity in the 2D systems. The MSD calculation used to determine the effective diffusivity only used the x-position of particles since the systems studied were much larger in $ \hat{x} $ than in $ \hat{y} $. This means that for some time steps, the calculated MSD was no different than the previous calculation since no movement occurred in $ \hat{x} $, but movement did occur in $ \hat{y} $. Second, simulation time and length needs to be mapped to physical time and length. Lastly, parameter space for the simulations needs to be explored further; the diffusive behaviours are well understood but the calculated effective diffusivity in the long-time and its exact dependence on different parameters mostly remains a mystery.